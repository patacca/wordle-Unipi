\section{Protocollo di comunicazione}
Il protocollo di comunicazione client-server e' stato implementato avendo come obiettivi:
\begin{itemize}
	\item La dimensione dei messaggi inviati
	\item La facilita' del parsing dei messaggi
	\item L'indipendenza del protocollo da qualsiasi implementazione, linguaggio di programmazione, sistema operativo o architettura utilizzata
\end{itemize}

\begin{lrbox}{\asciiart}
	\begin{varwidth}{\maxdimen}
		\noindent\lstinputlisting[basicstyle=\ttfamily]{proto_fmt1.txt}
	\end{varwidth}
\end{lrbox}%

E' stato dunque scelto un protocollo custom binario in cui i messaggi vengono codificati in \textbf{UTF-8} e incapsulati all'interno di un pacchetto (come si puo' vedere in Figura \ref{fig:proto_fmt1}) nel seguente formato:
\begin{enumerate}
	\item Dimensione in byte del messaggio codificato in 4 bytes $BIG\_ENDIAN$
	\item Il messaggio vero e proprio (codificato in \emph{UTF-8})
\end{enumerate}

\begin{center}
	\begin{figure}[t!]
		\makebox[\textwidth]{\showasciiart{80.5ex}}
		\caption{Rappresentazione dell'incapsulamento di un messaggio.}
		\label{fig:proto_fmt1}
	\end{figure}
\end{center}

Per una descrizione approfondita del formato dei messaggi consultare l'appendice \ref{appendix:format}