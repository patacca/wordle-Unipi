\section{Descrizione del progetto}
Il progetto consiste nell'implementazione (in Java 8) di una versione semplificata del famoso gioco \textbf{Wordle}. In particolare questa variante del gioco supporta numero di tentativi e lunghezza delle parole variabili\footnote{I valori sono impostati staticamente (\emph{hardcoded})}. \newline
Il progetto si compone di due parti indipendenti fra loro: il \textbf{server} e il \textbf{client}.

La struttura e' la seguente:
\bigskip

\dirtree{%
	.1 /.
	.2 client/ \quad \begin{minipage}[t]{7cm}
		Implementazione del client
	\end{minipage}.
	.3 backend/ \hfill \begin{minipage}[t]{7cm}
		Offre un'implementazione di base delle funzionalita'
		supportate dal client
			\end{minipage}.
	.4 exceptions/.
	.3 frontend/ \hfill \begin{minipage}[t]{7cm}
		Implementazione frontend
	\end{minipage}.
	.4 CLI/ \quad \begin{minipage}[t]{7cm}
		Command Line Interface
	\end{minipage}.
	.4 GUI/ \quad \begin{minipage}[t]{7cm}
		Graphical User Interface (Opzionale)
	\end{minipage}.
	.2 protocol/ \hfill \begin{minipage}[t]{7cm}
		Costanti e strutture dati condivise relative al
		protocollo di comunicazione client-server over TCP
	\end{minipage}.
	.2 rmi/ \quad \begin{minipage}[t]{7cm}
		Interfacce e strutture dati condivise relative
		a RMI
	\end{minipage}.
	.3 exceptions/.
	.2 server/ \hfill \begin{minipage}[t]{7cm}
		Implementazione del server
	\end{minipage}.
	.3 logging/.
	.2 utils/ \quad \begin{minipage}[t]{7cm}
		Utilities
	\end{minipage}.
}