\section{Manuale d'uso}

Per la compilazione e l'esecuzione e' richiesto \textbf{Java 8} o superiore.

Esistono diversi modi per compilare il progetto:
\begin{itemize}
	\item \hyperlink{makefile}{\textbf{Makefile}} (Linux, MacOS, Unix-like)
	\item \hyperlink{gradle}{\textbf{Gradle}} (All platforms)
	\item \hyperlink{manual}{\textbf{Manualmente}} (All platforms)
\end{itemize}

\hypertarget{makefile}{}
\subsection{Makefile}

\paragraph{Per compilare}
\begin{verbatim}
	$ make all
\end{verbatim}

\paragraph{Per creare i files \texttt{jar}}
\begin{verbatim}
	$ make jar
\end{verbatim}

Il risultato della compilazione sara' nella cartella \texttt{build/}

\bigskip
\bigskip
\hypertarget{gradle}{}
\subsection{Gradle}

\subsubsection{Unix}
Di seguito verra' descritta la procedura per sistemi Unix-like (Linux, MacOS, ...)

\paragraph{Per compilare}
\begin{verbatim}
	$ ./gradlew build
\end{verbatim}

Il risultato sara' nella cartella \texttt{app/build/}

\paragraph{Per creare i files \texttt{jar}}
\begin{verbatim}
	$ ./gradlew clientJar
	$ ./gradlew serverJar
\end{verbatim}

Il risultato sara' nella cartella \texttt{app/build/libs/}

\medskip
\subsubsection{Windows}
Di seguito verra' descritta la procedura per sistemi Windows

\paragraph{Per compilare}
\begin{verbatim}
	gradlew.bat build
\end{verbatim}

Il risultato sara' nella cartella \texttt{app\textbackslash build\textbackslash}

\paragraph{Per creare i files \texttt{jar}}
\begin{verbatim}
	gradlew.bat clientJar
	gradlew.bat serverJar
\end{verbatim}

Il risultato sara' nella cartella \texttt{app\textbackslash build\textbackslash libs\textbackslash}

\bigskip
\bigskip
\hypertarget{manual}{}
\subsection{Metodo manuale}

\underline{\textbf{Non raccomandato}}

\subsubsection{Unix}
Di seguito verra' descritta la procedura per sistemi Unix-like (Linux, MacOS, ...)

\paragraph{Per compilare}
\begin{verbatim}
	$ javac -cp lib/gson-2.10.1.jar:./app/src/main/java: -d ./build \ 
	    app/src/main/java/edu/riccardomori/wordle/server/ServerMain.java
	$ javac -cp lib/gson-2.10.1.jar:./app/src/main/java: -d ./build \
	    app/src/main/java/edu/riccardomori/wordle/client/ClientMain.java
\end{verbatim}

Il risultato sara' nella cartella \texttt{app/build/}

\medskip
\subsubsection{Windows}
Di seguito verra' descritta la procedura per sistemi Windows

\paragraph{Per compilare}
\begin{verbatim}
	javac.exe -cp lib\gson-2.10.1.jar;.\app\src\main\java; -d .\build \ 
	app\src\main\java\edu\riccardomori\wordle\server\ServerMain.java
	javac.exe -cp lib\gson-2.10.1.jar:.\app\src\main\java: -d .\build \
	app\src\main\java\edu\riccardomori\wordle\client\ClientMain.java
\end{verbatim}

Il risultato sara' nella cartella \texttt{app\textbackslash build\textbackslash}

\bigskip
\subsection{Esecuzione}

Per eseguire l'applicazione e' \underline{consigliato} usare i file \texttt{jar} precedentemente compilati nel seguente modo:
\begin{verbatim}
	$ java -jar client.jar
	$ java -jar server.jar
\end{verbatim}

Altrimenti e' possibile lanciare l'applicazione manualmente nel seguente modo:

\paragraph{Unix:}
\begin{verbatim}
	$ java -cp lib/gson-2.10.1.jar:./build/ \
	     edu.riccardomori.wordle.client.ClientMain
	$ java -cp lib/gson-2.10.1.jar:./build/ \
	     edu.riccardomori.wordle.server.ServerMain
\end{verbatim}


\paragraph{Windows:}
\begin{verbatim}
	java.exe -cp lib\gson-2.10.1.jar;.\build edu.riccardomori.wordle.client.ClientMain
	java.exe -cp lib\gson-2.10.1.jar;.\build edu.riccardomori.wordle.server.ServerMain
\end{verbatim}
